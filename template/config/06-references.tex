% ============================================================
% HKUST(GZ) Thesis - Bibliography Configuration
% ============================================================

% 加载 biblatex(深度定制以符合 IEEE 及学校要求)
\RequirePackage[
  backend=biber,
  style=ieee,           % 基础使用 IEEE 样式
  citestyle=numeric-comp,
  sorting=none,         % 按引用顺序
  natbib=true,
  hyperref=true,
  url=true,
  doi=true,
  eprint=true,
  isbn=false,
  giveninits=true,      % 名字缩写 (J. L. Doe)
  terseinits=false,     % 缩写后保留点 (J. L. 而不是 JL)
  maxnames=6,           % 超过6人使用 et al.
  minnames=1,
  dashed=false,         % 即使同一作者也不使用横线代替名字
]{biblatex}

% ------------------------------------------------------------
% 1. 格式修正:确保符合“Initials Lastname”要求
% ------------------------------------------------------------
% 强制姓名顺序为:首字母 + 姓 (e.g., J. L. Schonberger)
\DeclareNameAlias{default}{given-family}

% ------------------------------------------------------------
% 2. 标题与目录设置
% ------------------------------------------------------------
\DefineBibliographyStrings{english}{
  bibliography = {Bibliography and References},
  references = {Bibliography and References},
}

% 定义目录跳转和标题样式
\defbibheading{bibliography}[Bibliography and References]{%
  \chapter*{#1}%
  \addcontentsline{toc}{chapter}{#1}%
  \markboth{#1}{#1}%
}

% ------------------------------------------------------------
% 3. 解决溢出(Overfull \hbox)的“终极补丁”
% ------------------------------------------------------------
% 允许在 URL 和 DOI 中的任意位置断行
\setcounter{biburllcpenalty}{7000}
\setcounter{biburlucpenalty}{8000}
\setcounter{biburlnumpenalty}{9000}

% 针对参考文献列表的特殊排版微调
\renewcommand*{\bibfont}{\small\raggedright} % 1. 稍微调小字号 2. 强制左对齐避免溢出

% 允许在斜杠 (/) 后断行,专门对付 IEEE/CVF 这种词
\renewcommand*{\UrlBreaks}{\do\/\do-\do\_}

% ------------------------------------------------------------
% 4. 打印参考文献命令(集成 sloppy 避坑指南)
% ------------------------------------------------------------
\newcommand{\printthebibliography}[1][]{
  \clearpage
  \begingroup
    \sloppy          % 放宽间距限制,允许 LaTeX 更好地断行
    \printbibliography[heading=bibliography, #1]
  \endgroup
}

% ------------------------------------------------------------
% 5. 兼容性与细节微调
% ------------------------------------------------------------
\DeclareRobustCommand{\citep}[2][]{\cite[#1]{#2}}
\DeclareRobustCommand{\citet}[2][]{\textcite[#1]{#2}}

% 确保方括号内编号正确对齐
\DeclareFieldFormat{labelnumberwidth}{\mkbibbrackets{#1}}

% \AtBeginDocument{
%   \@ifpackageloaded{hyperref}{
%     \hypersetup{
%       hidelinks  % 直接隐藏所有超链接的颜色和方框
%     }
%   }{}{}
% }

% \renewbibmacro*{volume+number}{%
%   \printfield{volume}%         % 打印卷号
%   \setunit*{\addcomma\space}%  % 后面加逗号和空格
%   \printfield{number}%         % 打印期号
% }
