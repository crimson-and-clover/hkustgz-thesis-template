% ============================================================
% HKUST(GZ) Thesis - Math Configuration
% 数学环境配置:公式、定理等
% ============================================================

% 基础数学包
\RequirePackage{amsmath}
\RequirePackage{amssymb}
\RequirePackage{amsfonts}
\RequirePackage{amsthm}

% 数学符号增强
\RequirePackage{mathtools}     % amsmath 的扩展
\RequirePackage{bm}            % 粗体数学符号

% 定理环境定义
\theoremstyle{definition}
\newtheorem{definition}{Definition}[chapter]
\newtheorem{example}{Example}[chapter]
\newtheorem{assumption}{Assumption}[chapter]

\theoremstyle{plain}
\newtheorem{theorem}{Theorem}[chapter]
\newtheorem{lemma}{Lemma}[chapter]
\newtheorem{corollary}{Corollary}[chapter]
\newtheorem{proposition}{Proposition}[chapter]
\newtheorem{conjecture}{Conjecture}[chapter]

\theoremstyle{remark}
\newtheorem{remark}{Remark}[chapter]

% 证明环境
\renewcommand{\proofname}{Proof}

% 算法环境(伪代码)
\IfFileExists{algorithm2e.sty}{
  \RequirePackage[ruled,linesnumbered]{algorithm2e}
  \SetAlgoCaptionLayout{centering}
}{}

% 常用数学命令
\DeclareMathOperator*{\argmax}{arg\,max}
\DeclareMathOperator*{\argmin}{arg\,min}
\DeclareMathOperator{\sign}{sign}
\DeclareMathOperator{\diag}{diag}
\DeclareMathOperator{\tr}{tr}
\DeclareMathOperator{\rank}{rank}
\DeclareMathOperator{\var}{Var}
\DeclareMathOperator{\cov}{Cov}
\DeclareMathOperator{\E}{\mathbb{E}}

% 数集符号
\newcommand{\R}{\mathbb{R}}
\newcommand{\N}{\mathbb{N}}
\newcommand{\Z}{\mathbb{Z}}
\newcommand{\Q}{\mathbb{Q}}
\newcommand{\C}{\mathbb{C}}

% 向量/矩阵符号(粗体)
\newcommand{\vect}[1]{\boldsymbol{#1}}
\newcommand{\mat}[1]{\boldsymbol{#1}}

% 常用引用命令
\newcommand{\figref}[1]{Figure~\ref{#1}}
\newcommand{\tabref}[1]{Table~\ref{#1}}
\newcommand{\eqnref}[1]{Equation~\eqref{#1}}
\newcommand{\secref}[1]{Section~\ref{#1}}
\newcommand{\chapref}[1]{Chapter~\ref{#1}}
\newcommand{\algoref}[1]{Algorithm~\ref{#1}}
\newcommand{\thmref}[1]{Theorem~\ref{#1}}
\newcommand{\defref}[1]{Definition~\ref{#1}}

% 优化符号
\newcommand{\st}{\text{s.t.}}  % subject to
\newcommand{\subjectto}{\mathrel{\text{s.t.}}}

% 微分符号
\newcommand{\diff}{\mathrm{d}}
\newcommand{\pdiff}[2]{\frac{\partial #1}{\partial #2}}

% 范数和内积
\newcommand{\norm}[1]{\left\|#1\right\|}
\newcommand{\inner}[2]{\left\langle #1, #2 \right\rangle}

% 集合表示
\newcommand{\set}[1]{\left\{#1\right\}}

% 引用缩写
\newcommand{\etal}{\textit{et~al.}\xspace}
\newcommand{\eg}{\textit{e.g.},\xspace}
\newcommand{\ie}{\textit{i.e.},\xspace}
\newcommand{\wrt}{\textit{w.r.t.}\xspace}
\newcommand{\aka}{\textit{a.k.a.}\xspace}
\newcommand{\etc}{\textit{etc.}\xspace}
\newcommand{\cf}{\textit{cf.}\xspace}
