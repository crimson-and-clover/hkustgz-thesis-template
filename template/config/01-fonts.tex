% ============================================================
% HKUST(GZ) Thesis - Font Configuration
% 现代字体配置:使用 fontspec + xeCJK
% ============================================================

% 检查编译引擎
\RequirePackage{ifxetex}
\ifxetex
  % XeLaTeX 模式下使用 fontspec
  \RequirePackage{fontspec}
  
  % ============================================================
  % 西文字体设置
  % 优先级:1) 本地字体文件 2) 系统已安装字体 3) 开源替代字体
  % ============================================================
  
  % 检查本地字体文件是否存在(用于 CI/CD)
  \IfFileExists{template/fonts/times.ttf}{
    % 使用本地字体文件(Times New Roman 系列)
    \setmainfont{times}[
      Path = template/fonts/,
      Extension = .ttf,
      UprightFont = *,
      BoldFont = *bd,
      ItalicFont = *i,
      BoldItalicFont = *bi
    ]
    
    \setsansfont{arial}[
      Path = template/fonts/,
      Extension = .ttf,
      UprightFont = *,
      BoldFont = *bd,
      ItalicFont = *i,
      BoldItalicFont = *bi
    ]
    
    \setmonofont{cour}[
      Path = template/fonts/,
      Extension = .ttf,
      UprightFont = *,
      BoldFont = *bd,
      ItalicFont = *i,
      BoldItalicFont = *bi,
      Scale = 0.9
    ]
  }{
    % 本地字体不存在,检查系统字体或回退到开源替代
    \IfFontExistsTF{Times New Roman}{
      % 系统已安装 Times New Roman(Windows/macOS)
      \setmainfont{Times New Roman}
      \setsansfont{Arial}
      \setmonofont{Courier New}[Scale=0.9]
    }{
      \IfFontExistsTF{texgyretermes-regular.otf}{
        % 使用文件名加载,绕过系统 fontconfig,直接从 TeX Live 路径抓取
        \setmainfont{texgyretermes-regular.otf}[
          BoldFont       = texgyretermes-bold.otf,
          ItalicFont     = texgyretermes-italic.otf,
          BoldItalicFont = texgyretermes-bolditalic.otf
        ]
        \setsansfont{texgyreheros-regular.otf}[
          BoldFont       = texgyreheros-bold.otf,
          ItalicFont     = texgyreheros-italic.otf,
          BoldItalicFont = texgyreheros-bolditalic.otf
        ]
        \setmonofont{texgyrecursor-regular.otf}[
          BoldFont       = texgyrecursor-bold.otf,
          ItalicFont     = texgyrecursor-italic.otf,
          BoldItalicFont = texgyrecursor-bolditalic.otf,
          Scale          = 0.9
        ]
      }{
        % 最后的保底(通常不会走到这一步,因为 TeX Live 必带 Termes)
        \setmainfont{DejaVu Serif}
        \setsansfont{DejaVu Sans}
        \setmonofont{DejaVu Sans Mono}[Scale=0.9]
      }
    }
  }
  
  % 中文支持
  \RequirePackage{xeCJK}
  \xeCJKsetup{AutoFakeBold=false}
  
  % 中文字体设置(使用开源字体,可根据系统调整)
  % 优先级:Source Han > Noto CJK > SimSun > PingFang > AR PL UMing
  % 这样确保 CI/CD 环境(Linux)优先使用可用的开源字体
  
  \IfFontExistsTF{Source Han Serif SC}{
    % 优先使用思源宋体(如果已安装)
    \setCJKmainfont{Source Han Serif SC}
    \setCJKsansfont{Source Han Sans SC}
    \setCJKmonofont{Source Han Sans SC}
  }{
    \IfFontExistsTF{Noto Serif CJK SC}{
      % Linux CI 环境:使用 Noto CJK(fonts-noto-cjk 包提供)
      \setCJKmainfont{Noto Serif CJK SC}
      \setCJKsansfont{Noto Sans CJK SC}
      \setCJKmonofont{Noto Sans Mono CJK SC}
    }{
      \IfFontExistsTF{SimSun}{
        % Windows 系统
        \setCJKmainfont{SimSun}
        \setCJKsansfont{SimHei}
        \setCJKmonofont{FangSong}
      }{
        \IfFontExistsTF{PingFang SC}{
          % macOS 系统
          \setCJKmainfont{PingFang SC}
          \setCJKsansfont{Heiti TC}
          \setCJKmonofont{STFangsong}
        }{
          % 最后尝试 AR PL UMing(fonts-arphic-uming 包提供)
          \setCJKmainfont{AR PL UMing CN}
          \setCJKsansfont{AR PL UMing CN}
          \setCJKmonofont{AR PL UMing CN}
        }
      }
    }
  }
  
  % 标点符号处理
  \punctstyle{quanjiao}
  
  % 行距设置
  \RequirePackage{setspace}
  \RequirePackage{ulem}
  \onehalfspacing
  
\else
  % 非 XeLaTeX 模式下给出错误
  \ClassError{hkustgz-thesis}{This class requires XeLaTeX. Please compile with xelatex instead of pdflatex or lualatex.}{}
\fi

% 数学字体(可选:使用 unicode-math 统一数学字体)
% \RequirePackage{unicode-math}
% \setmathfont{Latin Modern Math}
