% ============================================================
% HKUST(GZ) Thesis - Page Layout Configuration
% 页面布局配置:使用 geometry 包
% ============================================================

% 页面尺寸和边距
\RequirePackage{geometry}
\geometry{
  a4paper,           % A4 纸张
  margin=25mm,       % 四周边距均为 25mm
  headheight=15pt,   % 页眉高度
  headsep=20pt,      % 页眉与正文间距
  footskip=30pt,     % 页脚与正文间距
  includemp=false,   % 不包含边注区域
  includehead=false,  % 包含页眉在正文区域内
  includefoot=false,  % 包含页脚在正文区域内
}

% 段落设置
\RequirePackage{indentfirst}  % 章节首段也缩进
\setlength{\parindent}{2em}   % 段落缩进 2 字符宽度
\setlength{\parskip}{0pt}     % 段落之间无额外间距

% 自动断行优化(使用 microtype 微排版)
\RequirePackage{microtype}    % 微排版优化,改善断行和间距

% 允许更宽松的行距调整,减少 Overfull \hbox 警告
\setlength{\emergencystretch}{3em}

% 行距设置(与 setspace 包配合)
% 已在 01-fonts.tex 中设置 \onehalfspacing

% 页眉页脚设置
\RequirePackage{fancyhdr}
\pagestyle{fancy}
\fancyhf{}  % 清空默认设置

% 页眉:左右都留空(页码在页脚居中)
\fancyhead[L]{}
\fancyhead[R]{}

% 页脚:页码居中(符合学校要求)
\fancyfoot[C]{\small\thepage}

% 章节首页使用 plain 样式(页脚居中页码)
\fancypagestyle{plain}{
  \fancyhf{}
  \fancyfoot[C]{\small\thepage}
  \renewcommand{\headrulewidth}{0pt}
}

% 页眉线宽度
\renewcommand{\headrulewidth}{0.4pt}

% 目录深度设置
\setcounter{tocdepth}{2}      % 目录显示到 subsection
\setcounter{secnumdepth}{2}   % 编号到 subsection
